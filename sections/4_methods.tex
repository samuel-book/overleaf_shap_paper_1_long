\renewcommand{\thefootnote}{\alph{footnote}} % Use letters for footnotes

\section{Methods}

All modelling an analysis was performed using Python in Jupyter Notebooks \cite{kluyver_jupyter_2016}, with general analysis and plotting performed using NumPy \cite{harris_array_2020}, Pandas \cite{mckinney-proc-scipy-2010}, Scikit-Learn  \cite{pedregosa_scikit-learn_2011}, and Matplotlib \cite{hunter_matplotlib_2007}. 

Further details of methods may also be found in the supplementary material. 

All code, with detailed results, used is available online at \url{https://samuel-book.github.io/samuel_shap_paper_1/}.

\subsection{Data}

Data was obtained from the Sentinel Stroke National Audit (SSNAP \footnote{https://www.strokeaudit.org/}). SSNAP has near-complete coverage of all acute stroke admissions in the UK (outside Scotland). All hospitals admitting acute stroke participate in the audit, and year-on-year comparison with Hospital Episode Statistics \footnote{https://digital.nhs.uk/data-and-information/data-tools-and-services/data-services/hospital-episode-statistics} confirms estimated case ascertainment of 95\% of coded cases of acute stroke.

The NHS Health Research Authority decision tool \footnote{http://www.hra-decisiontools.org.uk/research/} was used to confirm that ethical approval was not required to access the data. Data access was authorised by the Healthcare Quality Improvement Partnership (HQIP)\footnote{https://www.hqip.org.uk/} (reference HQIP303). 

Data was retrieved for 246,676 emergency stroke admissions to acute stroke teams in England and Wales between 2016 and 2018 (three full years). 88,928 patients arrived within 4 hours of known stroke onset.

Details of the data fields obtained are provided in the appendix.

\subsection{Machine learning models}

We used an \emph{Extreme Gradient Boosting model \cite{chen_xgboost_2016}} (`XGBoost`) to predict probability of use of thrombolysis for each patient. A single model was fitted for all hospitals, with hospital attended being one patient feature.

Features were selected from the 60 original features by forward-feature selection, selecting features one at a time by which feature gives most improvement in accuracy as measured by Receiver Operating Characteristic (ROC) Area Under Curve (AUC). Model accuracy was tested using 5-fold cross-validation.

\subsection{Shapley additive (SHAP)values}

SHAP provides a measure of the contribution of each patient feature value to the final predicted probability of receiving thrombolysis. For calculation of SHAP values we used the Shap library \cite{lundberg_unified_2017}.

\subsection{10k patient cohort}

As one method of comparing thrombolysis use between methods, we separated out 10k patients, chosen at random. The model was trained on the remaining patients. The predicted use of thrombolysis for the 10k cohort in each hospital was predicted by passing the same patients through the model, by changing hospital ID.

\subsection{Artificial patients}

As another method of comparing thrombolysis use between methods, we constructed artificial patients and passed those through the model for each hospital. When investigating artificial patients, we trained the model on all real patients. 