\section{Introduction}

% Include
% 1) What is the problem?
% 2) What do we know about low and varying use of thrombolysis
% 3) What do we not know
% 4) How are we addressing what we don't know

% 1) What is the problem?

Stroke remains one of the top three global causes of death and disability \cite{feigin_global_2021}. Despite reductions in age-standardised rates of stroke, ageing populations are driving an increase in the absolute number of strokes\cite{feigin_global_2021}. Across Europe, in 2017, stroke was found to cost healthcare systems \texteuro 27 billion, or 1.7\% of health expenditure \cite{luengo-fernandez_economic_2020}. Thrombolysis, using a recombinant form of tissue plasminogen activator, can significantly reduce the disability burden of ischaemic stroke, so long as it is given in the first few hours after stroke onset \cite{emberson_effect_2014}. Despite thrombolysis being of proven benefit in ischaemic stroke, use of thrombolysis varies significantly both between and within European countries \cite{aguiar_de_sousa_access_2019}. In England and Wales the national stroke audit reported that in 2021/22, thrombolysis rates for emergency stroke admissions varied from just 1\% to 28\% between hospitals, \cite{sentinel_national_stroke_audit_programme_ssnap_2022}, with a median rate of 10\% and an inter-quartile range of 8\%-13\%, against a 2019 NHS England long term plan that 20\% of patients of emergency stroke admissions should be receiving thrombolysis \cite{nhs_long_term_plan_2019}.

% 2) What do we know about low and varying use of thrombolysis

Studies have shown that reasons for low and varying thrombolysis rates are multi-factorial. Reasons include late presentation \cite{aguiar_de_sousa_access_2019}, lack of expertise \cite{aguiar_de_sousa_access_2019}, 
