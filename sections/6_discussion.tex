\section{Discussion}

Our machine learning model predicted use of thrombolysis with high accuracy, and also predicted the overall use of thrombolysis at each hospital with high accuracy. The model was well-calibrated, with the average predicted probability of thrombolysis for any group of patients matching the proportion of those patients who did receive thrombolysis. The SSNAP data used therefore appears to contain much of the information used to make thrombolysis decisions. To further support this, when we identified a group of patients that the model predicts would receive thrombolysis all the time in the large majority of hospitals, the observed thrombolysis rate of that group was 90\% in half the hospitals.

We do not intend our model for suggesting whether a patient \emph{should} receive thrombolysis. Despite our high accuracy, there will be sometimes be relevant patient details not known to us that would, and should, preclude use of thrombolysis. Rather we have used machine learning to determine the usual patterns of use of thrombolysis that are common across hospitals, and vary between hospitals. These learned patterns help isolate and show effects of varying patient features.

In general, using SHAP values to uncover the relationship between patient feature and the probability of receiving thrombolysis, we found that the probability of receiving thrombolysis fell with increasing arrival-to-scan times, was dependent on stroke severity (with the probability of receiving thrombolysis being highest between NIHSS 10 and 25), was lower when onset time was estimated rather than known precisely, and fell with increasing disability level before stroke. These patterns are similar to the observations of the discrete choice experiment with hypothetical patients \cite{de_brun_factors_2018}, but here we confirm these patterns in actual use of thrombolysis, can be quantitative about the effect, and we add the importance of time-to-scan and whether an onset time is known precisely. Using SHAP we can identify the general effect of each patient feature (such as stroke severity) and then also whether individual hospitals show a stronger or weaker response to changing values of that feature.

Hospital SHAP values correlated very closely with the predicted use of thrombolysis in a 10k cohort of patients, conforming that the hospital SHAP value provides a measure of the predisposition of a hospital to use thrombolysis. We found that hospital SHAP value explained 58\% of the variance in observed thrombolysis, suggesting that clinical decision-making around thrombolysis accounts for the majority of variance in use of thrombolysis between hospitals for those patients who arrive at hospital within 4 hours of known stroke onset time. 

After observing the general patterns that exit in the use of thrombolysis, we created subgroups of patients based reflecting what appeared to be an \emph{ideal} candidate for thrombolysis, and also for subgroups where we expected to see lower use of thrombolysis (low stroke severity, no precise onset time, pre-stroke disability). Observed thrombolysis in these groups reflected the patterns identified by the SHAP analysis. For the \emph{ideal} candidates of thrombolysis, half of stroke units would give thrombolysis to at least 90\% of these patients, but some units gave it to significantly fewer patients. Use of thrombolysis in the other subgroups of patients was, as expected, lower, but use also varied significantly, with use ranging between 0\% and 40\% for these less ideal patients. These patterns, both of average use, but also range of use between hospitals, were repeated with expected use of thrombolysis in the same 10k patient cohort of patients.

A final way of showing variation between units was to use artificial patients. We constructed a patient that was predicted to receive thrombolysis at almost every hospital, but when we changed reduced stroke severity and had an imprecise onset time the predicted use of thrombolysis dropped to 35\% of hospitals, but with a marked difference between \emph{benchmark} hospitals (the 30 hospitals with the highest predicted use of thrombolysis in the 10k cohort) where almost all hospitals would be predicted to give thrombolysis, and the non-benchmark hospitals where only one in five hospitals would be predicted to give thrombolysis.

Overall, use of thrombolysis has helped to elucidate patterns of thrombolysis in actual use, and also where hospitals vary in their attitudes to thrombolysis.

\subsection{Limitations}

This machine learning study is necessarily limited to data collected for the stroke national audit. Though we have high accuracy, and can identify clear patterns of use of thrombolysis, the data will not be sufficient to provide a decision-support tool. We may also be missing information that could otherwise have improved the accuracy even more. We also necessarily analyse thrombolysis at hospital level. Again we have high accuracy and can identify clear patterns, suggesting there  have a clear culture of use of thrombolysis, but we do not identify variation in thrombolysis between individual clinicians in the same hospital.


